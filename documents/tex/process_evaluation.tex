\pagestyle{empty}

\section{Process Evaluation\\{\small\tt{R.~Hales}}}
\subsection{Iteration One}
For our first coding iteration we chose to split the workload based on modules, so that each person would be able to individually work on their modules without having to rely on the others progress and allowing work to be done when we were not able to meet up. 
This process proved to have several problems. One of the major issues was the fact that when it was time to connect the modules together a fair amount of code had to be changed or replaced due to the modules being incompatible or because of unforeseen bugs. The other major problem that became clear is that not all of the group fully understood how every module worked due to not being involved in the coding process for some of the modules, this compounded with the first issue making it even more difficult to link the modules together.
\subsection{Iteration Two}
For the second iteration we re-evaluated our coding method and decided it would be best if we ensured the majority of our programming was done as a group,utilising pair programming. Having all code programmed  in the presence of and with the input of the whole group, helped solve the problems from iteration one,making it a lot easier to make sure all the modules are compatible and allowing all the group to understand how the code works.
 The other reason for this change was, due to the nature of the concurrent system, the code for the system had started to become too interlinked for individual modules to be changed without it affecting the entirety of the system. If we had carried on attempting to work on modules individually we would likely have ended up with multiple incompatible versions of every module which would then require more time and work to make compatible again. We believed it would be more efficient to try to keep multiple versions of the code to a minimum to prevent incompatibility.
\clearpage
\endinput


