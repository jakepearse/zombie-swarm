The software which we tenatively call 'the client' was designed to fullfill two major functions.
Provide a visulaisation of the current state of the simulation, as we were aware this would be the uer-facing component of the system we wanted to make it attractive and easy to interpret.
The secondary function of the software is to provide a daignostic tool to be used to examine the state of the system and individual entiys within it.
Early on the group seleceted HTML and JavaScript as the technologys which the client would leverage. The initial stages of development centred around establishing two-way communication via webscoket. The job of establishing the socket was delegated to the Cowboy framework (link to cowboy). For the sake of convience we decided on JSON as the format of websocket messages, on the Erlang side we initally identifed the MochiJSON library as our method for encodeing to JSON however as development progressed we ran into some difficulties with the library, Specifically relating to the encoding of atoms as strings and we switched to the JSX library for our JSON encoder.
The client software underwent one major revision, the inital version was constructed peice by peice in JavaScript in response to each change in the simulation arhectecture resulting in a lot of messy and difficult to maintain code, once the archetecture was largely implemented the client was entirely rewritten using the angularJS framework.

