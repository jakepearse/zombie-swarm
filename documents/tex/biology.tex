\pagestyle{empty}
%\setcounter{section}{2}

\title{Biology}

\section{Biology}
We want our simulation to be believable, as well as internally logically consistent,it should as much as possible correspond to known biological systems. The zombies must be alive, there are no examples of real parasitic infections which are able to make corpses walk around and bite people.

\subsection{Parasite-altered behaviour}
Although there are real examples of behaviour altering parasites causing the accelerated death of the host organism\footnote{Rabies.} parasitic infections which alter the hosts behaviour, do so to optimise the spread of the parasite\footnote{Liver fluke, Cholera} so it's not entirely unreasonable that the induced biting behaviour of Zombies serves two purposes;
\begin{enumerate}
\item To spread the infection.
\item To provide the host with nutrients.
\end{enumerate}

\subsection{Host mortality}
It would be counter-productive to kill the intended bitten host but perhaps the incubation of the Zombie infection causes a comatose state prior to symptomatic manifestation.

\subsection{Incubation}
Even the most aggressive viruses, such a particularly virulent strains of influenza have incubation periods. In fictional Zombie accounts we often see incubation periods ranging from days to almost immediate appearance of symptoms. Multicellular parasite's which head directly into an introduced hosts nervous system would therefore be a better explanation. Multicellular organisms such as trematodes (flukes) seem the most plausible explanation.

\subsection{Metabolic Changes}
A precedent for altering the host metabolism, explicitly by slowing it down is found in the Zombification of cockroaches in the Jewel Wasp's reproductive strategy where it can be speculated to serve the purpose of keeping the host alive for longer by conservation of energy.
In many Zombie fictions the parasite is already present in asymptomatic individuals and Zombie bites are not the vector of parasitic transmission. It's the loss of life that triggers symptomatic response, this is less easy to explain in a suitably plausible and robust way, but is important for the narrative of many Zombie accounts, It could tentatively be assumed that the parasites have a more complex life-cycle and an alternative primary infection vector. Therefore non-fatal injury could be the conditions under which symptoms manifest, as a last ditch effort by the parasite to spread already mature offspring.
Under such a conceit it's possible that non-symptomatic individuals are vectors for the spread of the parasite and it's only when the host organism's nominal biological functions are compromised that the parasite induces Zombie like symptoms.

\subsection{An approximation of zombie energy constrains}
There must be a balance between conservation of energy and spreading infection over as wide an area as possible.
\subsubsection{Movement}
A single human step uses 1/20th of a calorie\footnote{http://www.livestrong.com/article/320124-how-many-calories-does-the-average-person-use-per-step/} and measures approx 0.75m\footnote{http://www.thewalkingsite.com/10000steps.html} 
which equates to 15m/cal.
\subsubsection{Nutrition}
Raw pork (human analouge) contains 1.5 cal/g\footnote{http://caloriecount.about.com/calories-pork-i76775}
Lets assume you can remove 1kg of matter from a victim without killing them.
Also living organisms require fluid, humans can survive 3-5 pints of blood loss so we can assume our zombies get fluid from there too, blood also contains about 0.9 cal/ml.
A pint (uk) is about 568ml so that's 2556 extra calories per victim.
So in total a zombie victim provides 4056 calories give or take.
\subsubsection{Background metabolism}
We took an average BMR of 1927\footnote{average for a male in the U.S (86.092 * 13.75) + (182.8 * 5) - (6.76 * 35) + 66 = 1927.165} and quartered it \footnote{Assume that all non essential bodily functions are curtailed lowering the background calorie consumption to a quater.} to allow for the zombies slow metabolism giving us about 481 calories burned per day at rest.
\endinput
