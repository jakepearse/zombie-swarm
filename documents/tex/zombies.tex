
\pagestyle{empty}
\setcounter{section}{1}
\title{Zombie Lore}
\maketitle

\section{Types of Zombie}The background material on Zombies is broadly split into two types of explanations for the existence of zombie humans;
\begin{description}
\item[Supernatural] phenomenon are responsible for reanimating human corpses.\footnote{White Zombie (1932), The Evil Dead (1981)} These animated corpse zombies are created with black magic and respond to the wishes of their creator. This type of Zombie is therefore not acting entirely independently and of little use in our simulation. Similarly the Zombie of Haitian Vodoun mythology, although not actually dead are enslaved to a Houngan sourcerer.
\item[Biological] albeit fantastic explanations, where the existence of zombies is the result of some type of virus or parasite, details vary from fiction to fiction but by picking and choosing between these ideas we can construct a Zombie conceit which is at very least logically coherent.
\end{description}
The Zombies may be able to react differently to injury than non infected humans, ignoring pain. But a fatal wound massive blood loss or damage to vital organs will eventually the kill the host.

Zombies are attracted by sound and movement but are somehow detered from biting already infected individuals, in many accounts the Zombies use scent to search out prey.

The hypothosis of our simulation is that the attaction to movement and sound is what causes the Zombies to group together in swarms and this is what our simulation is designed to explore. The path-finding abilities of the zombies are a major consideration.
And the motivation of an individual zombie to select it's goal.

Since zombies are not noted for their self-preservation instincts their motivations can be broken down as;

Can I bite someone?  bite them.
Can I reliably detect someone to bite?  move towards to them.
Can I detect a suggestion of someone to bite?  move towards the stimulus.
\endinput
