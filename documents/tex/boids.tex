\pagestyle{empty}

\section{The Boids Model\\{\small\tt J.~Mitchard}}
After deciding that Particle Swarm Optimisation was not the most appropriate means of achieving our goal, we decided to implement an algorithm based on Boids.

\subsection{Boids}
\label{boids_document}
\subsubsection{What are Boids?}
The Boids model is an example of an individual-based flocking algorithm used to simulate the behaviour of large numbers of autonomous entities. First designed in 1986 by Craig Reynolds\footnote{http://www.red3d.com/cwr/boids/}, the model incorporates three simple steering behaviours in order to flock to other boids near itself.

\subsubsection{The Boids Model}
The Boids Model defines a number of behaviours that allow for different entities to flock together.
\begin{itemize}
	\item Steering behaviours
	\begin{itemize}
		\item \emph{Separation} - ensures that the entities will not overcrowd local flockmates. This prevents the flock converging on one entity and getting stuck in a localised swarm.
		\item \emph{Alignment} - allows the flock to maintain a heading, by aligning other members of the flock to navigate to an average heading of nearby entities.
		\item \emph{Cohesion} - the opposite of separation, this allows the independent entities to flock together by navigating towards other nearby entities.
	\end{itemize}
	\item Advanced navigation
	\begin{itemize}
		\item \emph{Collision Avoidance} - ensuring that the members of the flock do not bounce into other entities or obstacles. This is the ability to change heading instead of hitting an obstacle.
	\end{itemize}
	\item Attractors and Repulsors
	\begin{itemize}
		\item \emph{Flock Attraction} - an attraction to entities of similar type. This will be used by the Cohesion and Alignment steering behaviours.
		\item \emph{Super Attractors} - other entities considered higher priority to flock towards.
		\item \emph{Super Repulsors} - other entities considered a 'threat', flocks will prioritise moving away from these.
	\end{itemize}
\end{itemize}

\subsection{The Boids Algorithm}
This model will need some adapting to suit our own simulation. The model is intended to simulate the flocking of entities, however our project incorporates two separate types of entities that will flock for different reasons and have distinct behavioural rules separating them from each other.

\subsubsection{Boids In Context}
Zombies and humans will need to flock in different ways and for different reasons. We have decided that zombies are not inherently intelligent, and would tend to 'flock' through being attracted to movement and noise of other zombies. In a sense, a horde of zombies will flock together wandering reasonably aimlessly until a human entity enters the area. At this point the horde would shift behaviour and attempt to chase the human(s) that are in range.

This type of relation defines that a human, to a zombies is a Super Attractor. Super Attractors are prioritised higher than other behaviours, such as collision avoidance and flock coherency, therefore allowing our horde of zombies to demonstrate the type of behaviour displayed by zombies in popular culture of getting stuck against obstacles in an attempt to reach the human by the shortest path possible.

In order to simulate a realistic situation of zombies and humans, this would mean that the zombies must be a form of Super Repulsor to a human; meaning that they will attempt to run away from the zombies. Humans, however, are inherently more intelligent than the basic instinct driven zombies that we are implementing. Whilst trying to escape, they will not just pick a direction and run. Whilst the aim for the human entities would be to escape the horde of zombies chasing it, collision avoidance must still be carried out whilst pathfinding an escape route.

\clearpage
\endinput
