\pagestyle{empty}

\section{Updated Client Requirements Specification\\{\small\tt{J.~Mitchard,R.~Hales}}}

\subsection{The Web Client}
Now we have decided to implement the client using Javascript and D3, we can make our requirements specifications more specific to the target platform.

\subsection{What We Need}
This first section will explain the parts of the client that we feel are most critical to the workings of our system. 

We need the client to:
\begin{itemize}
	\item visually represents the current state of the simulation,
	\item display the continuous changing states within the system efficiently,
	\item be cross platform,
	\begin{itemize}
		\item this includes looking and functioning the same across different browsers,
		\item this includes avoiding browser specific tools and API's,
		\item this also means avoiding system specific API's such as Adobe Flash and Microsoft Silverlight that work differently or are not available on certain Operating Systems.
	\end{itemize}
	\item allow the user to start, reset and pause the simulation,
	\item allow the user to control certain parameters of the simulation,
	\item meet current web standards.
\end{itemize}

\subsection{What We Hope to Achieve}
We would like the client to:
\begin{itemize}
	\item visually represents the current state of the simulation in an aesthetically pleasing manner,
	\begin{itemize}
		\item we would like the visualisation to display as much information as possible, without overcomplicating what is shown to the user,
		\item because this is going to be a web based system, we think that scalability and efficiency are a higher priority than a graphically intensive visualisation.
	\end{itemize}
	\item seamlessly display the continuous changing states within the system efficiently,
	\item host the Erlang server, and allow use of the system across the Internet.
	\item provide intuitive controls for the user.
\end{itemize}



\clearpage
\endinput
