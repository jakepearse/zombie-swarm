\section{The Boids Model\\{\small\tt J.~Mitchard}}
After deciding that Partical Swarm Optimisation was not the most appropriate means of achievening our goal, we decided to implement an algorithm based on Boids.

\subsection{Boids}
\subsubsection{What are Boids?}
The Boids model is an example of an individual-based flocking algorithm used to simulate the behaviour of large numbers of autonomous entities. First designed in 1986 by Craig Reynolds, the model incorperates three simple steering behaviours in order to flock to other boids near itself.

\subsubsection{The Boids Model}
% more formal description of the Boids model
The Boids Model defines a number of behaviours that allow for different entities to flock together.

\subsection{The Boids Algorithm}
This model will need some adapting to suit our own simulation. The model is intended to simulate the flocking of entities, however our project incorperates two seperate types of entities that will flock for different reasons and have distinct behavioural rules seperating them from each other.

\subsubsection{Boids In Context}
Zombies and Humans will need to flock in different ways and for different reasons. We have decided that Zombies are not inherently inteligent, and would tend to 'flock' through being attracted to movement and noise of other Zombies. In a sense, a horde of Zombies will flock together wandering reasonably aimlessly until a Human entitiy enters the area. At this point the horde would shift behaviour and attempt to chase the Human(s) that are in range. This type of relation defines that a Human to a Zombie is a Super Attractor. Super Atractors intend to be prioritised higher than other behaviours, such as collision avoidance and flock coherancy, therefore allowing our horde of Zombies to demonstrate the type of behaviour displayed by Zombies in popular culture of getting stuck against obsticals in attempt to reach the Human by the shortest path possible.

In order to simulate a realistic situation of Zombies and Humans, this would mean that the Zombies must be a form of Super Repulsor to a Human; meaning that they will attempt to run away from the Zombies. Humans, however, are inherently more intelligent than the basic motor driven Zombies we are implementing. Whilst trying to escape, they will not just pick a direction and run. Whilst the aim for the Human entities would be to escape the horde of Zombies chasing it, collision avoidance must still be carried out when pathfinding an escape route. 


\subsubsection{Pseudocode}
