\pagestyle{empty}

\section{Risks\\{\small\tt R~Hayles}}
\subsection{Deadlines not met}
\begin{itemize}
\item If deadlines are not met then work on the project may be slowed or halted  until the work is caught up on.
\item If a deadline is not met the work should be completed as soon as possible, work plans and milestones should be reconsidered accounting for the setback and the reason for the failure to hit the deadline should be discussed.
\end{itemize}
\subsection{Group members unavailable due to illness or injury}
\begin{itemize}
\item If a group member is not available due to health issues work can fall behind and organising planning and meetings can become harder.
\item The members work should be reallocated if they are unable to work, milestones and deadlines should be reassessed, if there is a serious problem the member should seek concessions. If the member can't meet the group to discuss this issue discussions should be held online/via phone/etc.
\end{itemize}
\subsection{Work is lost due to technical}
\begin{itemize}
\item If files are lost then the work contained on those files will need redoing.
\item To avoid this happening all work should have at least one back up copy that can be restored after a fault.
\end{itemize}
\subsection{Concurrent simulation is not feasible}
\begin{itemize}
\item If we can't design a concurrent simulation then we will need a different way of designing the simulation.
\item We should have a backup non-concurrent design for the simulation we can implement in case our original design will not work.
\end{itemize}
\clearpage
\endinput
