\pagestyle{empty}
\section{Coding Standards\\{\small\tt R.~Hales}}
The rules listed in this document are to be used as a guide during code production for our project. This is to ensure the code produced is consistent and of a good quality.

\subsection{Versions}
When milestones are reached within the code, the version number should be incremented to make this clear. The version number should reflect which iteration of the project the code is from as well as which version within that iteration. A changelog should also be kept to make clear the differences between each version.
\subsection{Naming Conventions}
All Modules, Functions, Atoms,etc. should be given a short (1-2 words) name that is descriptive of it's purpose to improve the readability of the code.
\subsection{Formatting}
As far as possible all formatting should be kept internally consistent. This includes bracket placement, indentation distance and messages.
\subsection{Comments}
All Functions,States, etc. should be clearly commented to explain their purpose.
\subsection{OTP}
The OTP framework should be used for erlang code where applicable.
\subsection{Compiling}
All code should compile successfully without warnings or errors.
\clearpage
\endinput
