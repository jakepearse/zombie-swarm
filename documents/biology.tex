\documentclass[a4paper,12pt]{article}
\usepackage{times}

\pagestyle{empty}
\setcounter{section}{2}
\begin{document}

\title{The biology of zombies}
\author{J.~Pearse\\{\small\tt jp480@kent.ac.uk}}
\date{\today}

\maketitle

\section{Introduction}
We want our simulation to be belivable, as well as internally logically consistent,it should as much as possible corrispond to known biological systems. Firstly; The zombies must be alive, there are no examples of real parasitoid or parasite infections which are able to make corpses walk around.

\subsection{Biological motivation}
Although there are real examples of behaviour altering parasites causing the accelerated death of the host organism\footnote{Rabies.} parasitic infections which alter the hosts behaviour, do so to optimise the spread of the parasite\footnote{Liver fluke, Cholera} so it's not entirely unreasonable that the induced biting behaviour of Zombies serves two purposes;\begin{enumerate}
\item To spread the infection.
\item To provide the host with nutrients.
\end{enumerate}
\subsection{Host mortality}
It would be counter-productive to kill the intended bitten host but perhaps the incubation of the Zombie infection causes a comatose state prior to symptomatic manifestation.
This leads to another consideration, in allowing it to survive longer and spread the parasite over a wider range.
\subsection{Incubation}
Even the most aggressive viruses, such a particularly virulent strains of influeza have incubation periods, the time required for a critical mass of viral cells to be reached in the hosts system. In fictional Zombie accounts we often see incubation periods ranging from days to almost immediate appearance of symptoms. Multicellular parasite's which head directly into an introduced hosts nervous system would be a better explanation. Such multicellular organisms as trematodes (flukes) seem the most plausible explanation.
\subsection{Biological effeciency}
A precedent for altering the host metabolism, explicitly by slowing it down is found in the Zombification of cockroaches in the Jewel Wasp's reproductive strategy where it can be speculated to serve the purpose of keeping the host alive for longer by conservation of energy.
In many Zombie fictions the parasite is already present in asymptomatic individuals and Zombie bites are not the vector of parasitic transmission. It's the loss of life that triggers symptomatic response, this is less easy to explain in a suitably plausible and robust way, but is important for the narrative of many Zombie accounts, It could tentatively be assumed that the parasites have a more complex life-cycle and that with an alternative primary infection vector, therefore non-fatal injury could be the conditions under which symptoms manifest as a last ditch effort by the parasite to spread already mature offspring.
Under such a conciet it's possible that non-symptomatic individuals are vectors for the spread of the parascite and it's only when the host organism's nominal biological functions are comprimised that the parascite induces Zombie like symptoms.
There must be a balance between conservation of energy and searching for a victim,
\subsection{An approximation of zombie energy constrains}
A single human step uses 1/20th of a calorie\footnote{http://www.livestrong.com/article/320124-how-many-calories-does-the-average-person-use-per-step/} and measures approx 0.75m\footnote{http://www.thewalkingsite.com/10000steps.html} 
which equates to 15m/cal. Raw pork (human analouge) contains 1.5 cal/g\footnote{http://caloriecount.about.com/calories-pork-i76775}
Lets assume you can remove 1kg of matter from a victim without killing them.
Also living organisims require fluid, humans can survive 3-5 pints of blood loss so we can assume our zombies get fluid from there too, blood also contains about 0.9 cal/ml.
A pint (uk) is about 568ml so that's 2556 extra calories per victim.
So in total a zombie victim provides 4056 calories give or take.
Lets say we took an avergae BMR of 1927\footnote{(86.092 * 13.75) + (182.8 * 5) - (6.76 * 35) + 66 = 1927.165}. and quarter it\footnote{Assume that all non essential bodily functions are curtailed lowering the background calorie consumption to perhaps 1/4.} to allow for the zombies slow metabolism giving us  
about 481 calories burned per day at rest.
\end{document}
\endinput
